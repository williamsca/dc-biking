% Created 2021-08-25 Wed 20:03
% Intended LaTeX compiler: pdflatex
\documentclass[letterpaper]{article}
\usepackage[utf8]{inputenc}
\usepackage[T1]{fontenc}
\usepackage{graphicx}
\usepackage{grffile}
\usepackage{longtable}
\usepackage{wrapfig}
\usepackage{rotating}
\usepackage[normalem]{ulem}
\usepackage{amsmath}
\usepackage{textcomp}
\usepackage{amssymb}
\usepackage{capt-of}
\usepackage{hyperref}
\author{Colin Williams}
\date{\today}
\title{Bikeshares}
\hypersetup{
 pdfauthor={Colin Williams},
 pdftitle={Bikeshares},
 pdfkeywords={},
 pdfsubject={},
 pdfcreator={Emacs 26.3 (Org mode 9.1.9)}, 
 pdflang={English}}
\begin{document}

\maketitle

\section{Introduction}
\label{sec:orgf3ff886}

City governments have invested in building out networks of bikeshare stations. These programs are popular, but nevertheless they incur substantial costs.

Prior work has found a positive effect of bikeshare stations on local rents\ldots{} 

These papers estimate the effect of connecting a neighborhood to the entire bikeshare network. However, they do not capture a key quantity of interest to the policy maker: the marginal value of an additional station at a particular location. This value will be composed of both a local effect of connecting the immediate neighborhood to the bikeshare network plus the global value of a more connections across the city.
\end{document}
